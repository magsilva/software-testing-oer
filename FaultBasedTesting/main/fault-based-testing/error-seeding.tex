\begin{frame}[parent={concept:fault-based-testing},hasprev=false,hasnext=true]
\frametitle{Error seeding}
\label{concept:error-seeding}

\begin{block:concept}{Error seeding}
Error seeding introduces a known number of artificial faults in the program
under testing before it is tested.
\end{block:concept}

\begin{block:fact}{Rationale}
\begin{itemize}
	\item Error seeding is based on the assumption that faults are uniformly
	distributed in the program.
\end{itemize}
\end{block:fact}
\end{frame}


\begin{frame}[hasprev=true,hasnext=true]
\frametitle{Error seeding}

\begin{block:fact}{How it works}
\begin{itemize}
	\item After testing using the error seeding criterion, it is possible to
	estimate the number of remaining natural faults from the total number of
	faults found and the rate between natural and artificial faults.
\end{itemize}
\end{block:fact}
\end{frame}



\begin{frame}[hasprev=true,hasnext=false]
\frametitle{Error seeding}

\begin{block:fact}{Limitations}
\begin{itemize}
	\item Error seeding requires programs that can support 10,000 faults or
	more in order to obtain a statistically reliable result.

	\item The artificial faults generated by error seeding may hide natural
	faults.
	\begin{itemize}
		\item This is in general is not the case, as real programs present
		long pieces of simple code with few faults and small pieces of high
		complexity and with many faults.
	\end{itemize}
\end{itemize}
\end{block:fact}
\end{frame}