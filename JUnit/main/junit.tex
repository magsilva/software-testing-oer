\begin{frame}[c, parent={cmap:software-testing}, hasprev=false, hasnext=false]
\frametitle{JUnit}
\label{cmap:junit}

\insertcmap{Courses-SoftwareTesting-JUnit}
\end{frame}


\begin{frame}[parent={cmap:junit}, hasprev=false, hasnext=true]
\frametitle{JUnit}
\label{concept:junit}

\begin{block:concept}{What is it?}
JUnit is an open-source framework to provide support for documenting and
automating the execution of test sets for Java programs.
\end{block:concept}


\begin{block:fact}{General information}
\begin{itemize}
	\item Developed by Kent Beck and Erich Gamma (in 1994).

	\item Hosted at \url{http://www.junit.org/} and
	\url{http://sf.net/projects/junit/}.
\end{itemize}
\end{block:fact}


\begin{block:fact}{Features}
\begin{itemize}
	\item Test cases implemented using annotations.

	\item Useful assertions collection.

	\item Fixtures enhances the design of test sets.
\end{itemize}
\end{block:fact}

\hfill
\refie{example:identifier-testcases-junit}{\beamerbutton{Example: Identifier}}
\end{frame}



\begin{frame}[hasprev=true, hasnext=true]
\frametitle{JUnit}
\framesubtitle{Installation}
\label{procedure:junit:installation}


\begin{block:fact}{Requirements}
\begin{itemize}
	\item JUnit requires the Java Development Kit version 1.5 or newer.
\end{itemize}
\end{block:fact}

\begin{block:procedure}{Download}
\begin{enumerate}
	\item Download JUnit at \url{http://sourceforge.net/projects/junit/}.
	\begin{itemize}
		\item Current version is 4.8.1.

		\item The application is distributed as a JAR file (comprised of just
		the JUnit library) and a compressed ZIP file (with the JUnit library
		and documentation).

		\item Download the ZIP file.
	\end{itemize}

	\item Uncompress the file on a given directory that you have written
	permission.
\end{enumerate}
\end{block:procedure}
\end{frame}


\begin{frame}[fragile]
\frametitle{JUnit}
\framesubtitle{Configuration}
\label{procedure:junit:configuration}

\begin{block:fact}{How to run it?}
\begin{itemize}
	\item To execute the JUnit application, you must add the JUnit library
	(junit-4.8.1.jar) to the Java Classpath.
\end{itemize}
\end{block:fact}

\begin{block:fact}{Classpath configuration}
\begin{itemize}
	\item You can add the library to the CLASSPATH environment variable.
\begin{lstlisting}
Unix: \srccode{export CLASSPATH=/opt/junit-4.8.1/junit-4.8.1.jar:$CLASSPATH}
Windows: \srccode{set CLASSPATH=C:\junit-4.8.1\junit-4.8.1.jar;%CLASSPATH%}
\end{lstlisting}

	\item You can use the -cp option when running the tests. This is the
	recommended option!
\begin{lstlisting}
java -cp /opt/junit-4.8.1/junit-4.8.1.jar <program>
\end{lstlisting}
\end{itemize}
\end{block:fact}
\end{frame}



\begin{frame}[hasprev=true, hasnext=false]
\frametitle{JUnit}
\framesubtitle{Shake down}
\label{procedure:junit:shakedown}


\begin{block:fact}{Is it working?}
\begin{itemize}
	\item To check whether JUnit was correctly installed, you can run the JUnit
	test suite.
	\begin{itemize}
		\item The class with all the test cases for JUnit is
		\srccode{org.junit.tests.AllTests}.

		\item This class is located at the root of JUnit installation directory.
	\end{itemize}
\end{itemize}
\end{block:fact}


\hfill
\refie{example:junit-shakedown}{\beamerbutton{Example: JUnit shakedown}}
\end{frame}

