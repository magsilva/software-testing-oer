\begin{frame}[parent={concept:junit}, hasprev=false, hasnext=false]
\frametitle{Fixture}
\label{concept:junit-fixture}
\label{concept:fixture}

\begin{block:concept}{Fixture}
\begin{itemize}
	\item Fixtures are actions that should be executed before or after a test
	case (usually to set up pre-conditions).

	\item It defines a fixed state of a set of objects used as a baseline for
	running tests.
\end{itemize}
\end{block:concept}

\begin{block:fact}{Why should I use fixtures?}
\begin{itemize}
	\item The purpose of a test fixture is to ensure that there is a well known
	and fixed environment in which tests are run so that results are
	repeatable.
\end{itemize}
\end{block:fact}
\end{frame}