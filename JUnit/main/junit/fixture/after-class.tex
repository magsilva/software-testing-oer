\begin{frame}[parent={concept:fixture}, hasprev=false, hasnext=false]
\frametitle{Fixture}
\framesubtitle{AfterClass}
\label{concept:junit-afterclass-fixture}
\label{concept:afterclass-fixture}
\label{concept:afterclass}

\begin{block:concept}{AfterClass}
AfterClass is a fixture that is used to cleanup modifications made
for or by a test set.
\end{block:concept}


\begin{block:fact}{How to use it?}
\begin{itemize}
	\item The AfterClass fixture is created by annotating a method with
	\srccode{@AfterClass}.

	\item AfterClass fixtures run after all the JUnit test cases in a class
	have been run.

	\item AfterClass fixtures declared in the superclasses will be run after
	those of the current class.

	\item No other ordering is defined when running AfterClass fixtures
	declared in the same class.
\end{itemize}
\end{block:fact}
\end{frame}