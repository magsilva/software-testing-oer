\begin{frame}[parent={cmap:jumble}, hasprev=false, hasnext=true]
\frametitle{Jumble}

\begin{block:concept}{Jumble}
Jumble is a mutation testing tool. It mutates Java bytecode and tests it
with JUnit test cases and suites.
\end{block:concept}

\begin{block:fact}{General information}
\begin{itemize}
	\item Available at \url{http://jumble.sourceforge.net}.
\end{itemize}
\end{block:fact}
\end{frame}


\begin{frame}[hasprev=true, hasnext=true]
\frametitle{Jumble}
\framesubtitle{Features}

\begin{block:fact}{Features}
\begin{itemize}
	\item A single mutation is performed for each test case.

	\item The following mutation operators are supported:
	\begin{itemize}
		\item conditionals,
		\item binary arithmetic operations,
		\item increments,
		\item inline constants,
		\item class pool constants,
		\item return values,
		\item switch statements.
	\end{itemize}
\end{itemize}
\end{block:fact}
\end{frame}


\begin{frame}
\frametitle{Jumble}
\framesubtitle{Installation}

\begin{block:procedure}{Installation}
\begin{enumerate}
	\item Download Jumble from \url{http://jumble.sf.net} and unpack it.
\end{enumerate}
\end{block:procedure}
\end{frame}


\begin{frame}
\frametitle{Jumble}
\framesubtitle{Test session configuration}

\begin{block:fact}{Command line}
\begin{itemize}
	\item Jumble can be executed using the following command:
	\begin{itemize}
		\item \srccode{java -jar jumble.jar <options> <parameters>}
	\end{itemize}
\end{itemize}
\end{block:fact}

\begin{block:fact}{Command line options}
\begin{itemize}
	\item \srccode{--classpath}: Defines the classpath required to run the
	application under testing.
	\item Options related to mutation operators:
	\begin{itemize}
		\item \srccode{--cpool}: Mutate constant pool entries.
		\item \srccode{--increments}: Mutate increments.
		\item \srccode{--inline-consts}: Mutate inline constants.
		\item \srccode{--return-vals}: Mutate return values.
		\item \srccode{--stores}: Mutate assignments.
		\item \srccode{--switch}: Mutate switch cases
	\end{itemize}
\end{itemize}
\end{block:fact}
\end{frame}


\begin{frame}[fragile]
\frametitle{Jumble}
\framesubtitle{Test session configuration}

\begin{block:fact}{Command line arguments}
\begin{itemize}
	\item The first argument is the class to be tested (full qualified name
	of the class).
	\item The second argument is a set of classes with JUnit test cases.
\end{itemize}
\end{block:fact}

\begin{block}{Example}
\begin{lstlisting}
java -jar jumble.jar --classpath=. example/Mover example/MoverTest
\end{lstlisting}
\end{block}
\end{frame}