\begin{frame}[parent={concept:mujava}, hasprev=false, hasnext=true]
\frametitle{muJava}
\framesubtitle{Installation}

\begin{block:procedure}{Installation}
\begin{enumerate}
	\item Save muJava to a directory (for example, \srccode{/opt/mujava-3.0}).

	\item Add muJava libraries (\srccode{mujava.jar} and \srccode{openjava2005.jar})
	to the \srccode{CLASSPATH}.

	\item Add the library \srccode{tools.jar}, distributed along Java SDK
	(usually at \srccode{$JAVA_HOME/lib/tools.jar}), to the CLASSPATH.
\end{enumerate}
\end{block:procedure}
\end{frame}


\begin{frame}[hasprev=true, hasnext=true]
\frametitle{muJava}
\framesubtitle{Test project configuration}

\begin{block:procedure}{Test project configuration}
\begin{enumerate}
	\item Configure the file \srccode{mujava.config} with the directory
	(absolute path) that MuJava will use for the test project data (for
	example, \srccode{MuJava_HOME=/opt/mujava-3.0}).
	\begin{itemize}
		\item This file must be accessible from the \srccode{CLASSPATH}.

		\item The value of \srccode{MuJava_HOME} must not have a trailing
		slash (otherwise muJava will not work correctly).
	\end{itemize}

	\item Create the following directories at \srccode{MuJava_HOME}:
	\srccode{classes}, \srccode{result}, \srccode{src}, and \srccode{testset}.

	\item Copy the source code of the application under testing to the
	directory \srccode{$MuJava_HOME/src}.

	\item Copy the compiled classes of the application under testing to the
	directory \srccode{$MuJava_HOME/classes}.

	\item Copy the test cases to the directory \srccode{$MuJava_HOME/classes}.
\end{enumerate}
\end{block:procedure}
\end{frame}



\begin{frame}
\frametitle{muJava}
\frametitle{Mutation generation}

\begin{block:procedure}{Mutant generation}
\begin{enumerate}
	\item Add the directory \srccode{$MuJava_HOME/classes} to the
	\srccode{CLASSPATH}.

	\item Run the command \srccode{java mujava.gui.GenMutantsMain}.

	\item Select the files to be tested (or click the button labeled
	\srccode{All} in the bottom left of the window).

	\item Select the mutation operators to be used to generate the mutants.

	\item Generate the mutants (using the yellow button in the bottom center
	of the window).
\end{enumerate}
\end{block:procedure}
\end{frame}


\begin{frame}
\frametitle{muJava}
\framesubtitle{Test case definition}

\begin{block:fact}{Test case format}
\begin{itemize}
	\item Test cases must comprise of public classes with public methods.

	\item The public methods represents each test case.
	\begin{itemize}
		\item The method name must start with \srccode{test}.
	\end{itemize}

	\item The methods must return some value (such as \srccode{String}).
	\begin{itemize}
		\item They cannot return \srccode{void}.
	\end{itemize}
\end{itemize}
\end{block:fact}


\begin{block:procedure}{Test case definition}
\begin{enumerate}
	\item Implement the test cases.
	\begin{itemize}
		\item Save them to \srccode{$MuJava_HOME/testset}.
	\end{itemize}

	\item Compile the test cases.
\end{enumerate}
\end{block:procedure}
\end{frame}


\begin{frame}
\frametitle{muJava}
\framesubtitle{Test case execution}

\begin{block:procedure}{Test case execution}
\begin{enumerate}
	\item Remove the directory \srccode{$MuJava_HOME/classes} of the
	\srccode{CLASSPATH}.

	\item Run the command \srccode{java mujava.gui.RunTestMain}.

	\item Select the class to be tested.
	\begin{itemize}
		\item Even if the correct class is already selected, click on the combo
		box and select it again, otherwise muJava will not recognize the
		mutants previously generated.
	\end{itemize}

	\item Select the the type of mutants to be executed.
	\begin{itemize}
		\item If no class mutants has been generated, select just the option
		\srccode{Execute only traditional mutants}, otherwise muJava will not
		run the test cases.
	\end{itemize}

	\item Run the test cases by activating the yellow button (which is labeled
	\srccode{RUN}).
\end{enumerate}
\end{block:procedure}
\end{frame}