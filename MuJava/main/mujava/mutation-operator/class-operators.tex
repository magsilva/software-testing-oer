\begin{frame}[parent={concept:mujava}, hasprev=false, hasnext=true]
\frametitle{Class operators}

\begin{block:concept}{Class operators}
Class-level mutation operators apply simple syntactical modifications to the
classes of the application under testing.
\end{block:concept}

\begin{block:fact}{Affected elements}
\begin{itemize}
	\item Based on language features related to object orientation, four
	groups of operatores are defined:
	\begin{itemize}
		\item Encapsulation.
		\item Inheritance.
		\item Polymorphism.
		\item Java-specific features.
	\end{itemize}
\end{itemize}
\end{block:fact}
\end{frame}



\begin{frame}[hasprev=true, hasnext=true]
\frametitle{Class operators}
\framesubtitle{Encapsulation}


\begin{block:fact}{Rationale}
\begin{itemize}
	\item It exploits the following facts:
	\begin{itemize}
		\item the semantics of the various access levels are often poorly
		understood,

		\item the access for variables and methods is often not considered
		during design.
	\end{itemize}

	\item Although poor access definition do not always cause faults, it can
	lead to faulty behaviour when the class is integrated with other classes.
\end{itemize}
\end{block:fact}

\begin{block:fact}{Operators}
\begin{itemize}
	\item \textbf{AMC} (Access modifier change): it changes the access level
	for instance variables and methods.
\end{itemize}
\end{block:fact}
\end{frame}


\begin{frame}
\frametitle{Class operators}
\framesubtitle{Inheritance}


\begin{block:fact}{Rationale}
\begin{itemize}
	\item Incorrect use of inheritance can lead to faults (variable shadowing,
	method overriding, parent access and constructors).
\end{itemize}
\end{block:fact}

\begin{block:fact}{Operators}
\begin{itemize}
	\item Variable shadowing:
	\begin{itemize}
		\item Hiding variable deletion (\textbf{IHD}) and insertion
		(\textbf{IHI}).
	\end{itemize}

	\item Method overriding:
	\begin{itemize}
		\item Overriding method deletion (\textbf{IOD}), change in position
		(\textbf{IOP}), and renaming (\textbf{IOR}).
	\end{itemize}

	\item Parent access:
	\begin{itemize}
		\item \srccode{super} keyword insertion (\textbf{ISI}) and deletion
		(\textbf{ISD}).
	\end{itemize}

	\item Constructor:
	\begin{itemize}
		\item Explict call to a parent's constructor deletion (\textbf{IPC}).
	\end{itemize}
\end{itemize}
\end{block:fact}
\end{frame}


\begin{frame}
\frametitle{Class operators}
\framesubtitle{Polymorphism}


\begin{block:fact}{Rationale}
\begin{itemize}
	\item Polymorphism allows the late binding of types to the object that will
	be actually accessed.
\end{itemize}
\end{block:fact}

\begin{block:fact}{Operators}
\begin{itemize}
	\item Create a new object using a child type (\textbf{PNC}).
	\item Declare a variable using the parent class type (\textbf{PMD}).
	\item Declare parameter using the parent class type (\textbf{PPD}).
	\item Type cast operator insertion (\textbf{PCI}), delection (\textbf{PCD}),
	and change (\textbf{PCC}).
	\item Reference assignment with other compatible type (\textbf{PRV}).
	\item Overloading method contents modification (\textbf{OMR}), and deletion
	(\textbf{OMD}).
	\item Overloading argument change (\textbf{OAC}).
\end{itemize}
\end{block:fact}
\end{frame}


\begin{frame}[hasprev=true, hasnext=false]
\frametitle{Class operators}
\framesubtitle{Java-specific features}


\begin{block:fact}{Types of expression operators}
\begin{itemize}
	\item Some mistakes are not due to intrinsic object-orientation features,
	but of Java-specific ones.
\end{itemize}
\end{block:fact}

\begin{block:fact}{Operators}
\begin{itemize}
	\item \srccode{this} keyword insertion (\textbf{JTI}) and deletion
	\textbf{JTD}).
	\item \srccode{static} keyword insertion (\textbf{JSI}) and deletion
	(\textbf{JSD}).
	\item Member variable initialization deletion (\textbf{JID}).
	\item Java-supported default constructor creation (\textbf{JDC}).
	\item Reference assignment and content assignment replacement
	(\textbf{EOA}).
	\item Reference comparison and content comparison replacement
	(\textbf{EOC}).
	\item Accessor method change (\textbf{EAM}).
	\item Modifierr method change (\textbf{EMM}).
\end{itemize}
\end{block:fact}
\end{frame}