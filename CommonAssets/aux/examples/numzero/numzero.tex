\begin{frame}[hasprev=false, hasnext=true, fragile]
\label{example:numzero}
\frametitle{numZero}
\framesubtitle{Description}

\begin{itemize}
	\item Is there any fault in this program?
\end{itemize}

\begin{lstlisting}
class numZero
{
    /**
     * If arr is null throw NullPointerException, else
     * return the number of occurrences of zero in arr.
     */
    public static int numZero (int[] arr)
    {
        int count = 0;
        for (int i = 1; i < arr.length; i++) {
            if (arr[i] == 0) {
                count++;
            }
        }
        return count;
    }
}
\end{lstlisting}
\end{frame}


\begin{frame}[hasprev=true, hasnext=true, fragile]
\frametitle{numZero}
\framesubtitle{Diagnostic}

\begin{itemize}
	\item The fault in this program is that it starts looking for zeroes at
	index 1 instead of index 0, as is necessary for arrays in Java.
	\begin{itemize}
		\item For example, numZero([2, 7, 0]) correctly evaluates to 1, while
		numZero([0, 7, 2]) incorrectly evaluates to 0.
		\begin{itemize}
			\item In both of these cases the fault is executed.

			\item Although both of these cases result in an error, only the
			second case results in failure.
		\end{itemize}
	 \end{itemize}
\end{itemize}

\begin{lstlisting}
        for (int i = 1; i < arr.length; i++) {
            if (arr[i] == 0) {
                count++;
            }
        }
        return count;
\end{lstlisting}
\end{frame}


\begin{frame}[fragile]
\frametitle{numZero}
\framesubtitle{Diagnostic}

\begin{itemize}
	\item For the first example given above, the state at the if statement on
	the very first iteration of the loop is (x = [2, 7, 0], count = 0, i = 1,
	PC = if).

	\item This state is in error precisely because the value of i should be
	zero on the first iteration.

	\item However, since the value of count is coincidentally correct, the
	error state does not propagate to the output, and hence the software does
	not fail.
\end{itemize}

\begin{lstlisting}
        for (int i = 1; i < arr.length; i++) {
            if (arr[i] == 0) {
                count++;
            }
        }
        return count;
\end{lstlisting}
\end{frame}


\begin{frame}[hasprev=true, hasnext=false, fragile]
\frametitle{numZero}
\framesubtitle{Diagnostic}

\begin{itemize}
	\item For the second example given above, the corresponding error state is
	(x = [0, 7, 2], count = 0, i = 1, PC = if).

	\item In this case, the error propagates to the variable count and is
	present in the return value of the method. Hence a failure results.
\end{itemize}

\begin{lstlisting}
        for (int i = 1; i < arr.length; i++) {
            if (arr[i] == 0) {
                count++;
            }
        }
        return count;
\end{lstlisting}
\end{frame}


