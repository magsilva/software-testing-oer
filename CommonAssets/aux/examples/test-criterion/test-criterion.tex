\begin{frame}[hasprev=false, hasnext=true]
\label{example:test-criterion}
\frametitle{Test criterion example}

\begin{itemize}
	\item Suppose we are given the enviable task of testing bags of jelly beans.
	We need to come up with ways to sample from the bags.

	\item Suppose these jelly beans have the following six flavors and come in
	four colors: Lemon (colored Yellow), Pistachio (Green), Cantaloupe (Orange),
	Pear (White), Tangerine (also Orange), and Apricot (also Yellow).

	\item A simple approach to testing might be to test one jelly bean of each
	flavor. Then we have six test requirements, one for each flavor.

	\item We satisfy the test requirement ``Lemon'' by selecting and, of course,
	tasting a Lemon jelly bean from a bag of jelly beans.
\end{itemize}
\end{frame}


\begin{frame}[hasprev=true, hasnext=false]
\frametitle{Test criterion example}

\begin{itemize}
	\item The ``flavor criterion'' yields a simple strategy for selecting jelly
	beans.

	\item In this case, the set of test requirements, $TR$, can be formally
	written out as \foreign{TR = \{flavor = Lemon, flavor = Pistachio,
	flavor = Cantaloupe, flavor = Pear, flavor = Tangerine, flavor = Apricot\}}.
\end{itemize}
\end{frame}


