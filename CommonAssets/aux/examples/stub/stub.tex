\begin{frame}[hasprev=false, hasnext=true]
\frametitle{Stub/Mock using Mockito}
\label{example:stub}

\begin{block:fact}{Mockito}
\begin{itemize}
	\item Mockito is a mocking framework.
	\begin{itemize}
		\item Mock is something that mimics an unit of software.

		\item A stub mocks an unit of software.
	\end{itemize}
\end{itemize}
\end{block:fact}

\begin{block:fact}{How does it work}
\begin{itemize}
	\item A mock is created for a given class.
	\begin{itemize}
		\item This mock is an instance of the given class, but which behaviour
		is undefined.
	\end{itemize}

	\item After the instantiation of the mock, the behaviour of every method
	that will be tested must be defined.
	\begin{itemize}
		\item For instance, you can define that the method \srccode{get}, when
		using the parameter \srccode{35}, will return the string
		\srccode{Tomato}.
	\end{itemize}

	\item Thus, complex objects do not need to be implemented: just stub the
	methods used in the test case.
\end{itemize}
\end{block:fact}
\end{frame}


\begin{frame}[hasprev=true, hasnext=true]
\frametitle{Stub/Mock using Mockito}
\framesubtitle{Example}

\begin{block}{Test case}
\begin{itemize}
	\item Imagine the following situation. The unit under testing is the class
	\srccode{IdentityMethodSet}.
	\begin{itemize}
		\item This class defines a set that uses as identity the method which
		name is given as parameter in the constructor (which is
		\srccode{toString} in the example).
	\end{itemize}

	\item You need to test the behaviour of the \srccode{IdentityMethodSet} for
	a class that can throw an exception when the method assigned as identity
	is used.
	\begin{itemize}
		\item \srccode{IdentityMethodSet} must throw an
		\srccode{UnsupportedOperationException} if it fails to use the identity
		method of the object.
	\end{itemize}
\end{itemize}
\end{block}
\end{frame}



\begin{frame}[hasprev=true, hasnext=true]
\frametitle{Stub/Mock using Mockito}
\framesubtitle{Example}

\begin{block}{Test case}
\begin{itemize}
	\item However, it may be difficult to find a class that implements such
	behaviour.

	\item So, instead of finding a class that has such behaviour, we just
	create a mock that stubs the method \srccode{toString}.
	\begin{itemize}
		\item And we configure this method, in the mock, to throw an exception.
	\end{itemize}
\end{itemize}
\end{block}
\end{frame}


\begin{frame}[hasprev=true, hasnext=false, fragile]
\frametitle{Stub/Mock using Mockito}
\framesubtitle{Example}


\begin{block}{Implementation}
\begin{lstlisting}[language=java]
import static org.mockito.Mockito.*;

public class IdentityMethodSetTest {
{
	@Test(expected=UnsupportedOperationException.class)
	public void testNew_InvalidIdentityMethod_UnsupportedOperation()
	{
		InetAddress clazz = mock(InetAddress.class);
		when(clazz.toString()).thenThrow(new NullPointerException());
		IdentityMethodSet set = new IdentityMethodSet<String, InetAddress>("toString");
		set.add(clazz);
	}
}
\end{lstlisting}
\end{block}

\begin{block}{Implementation details}
\begin{itemize}
	\item The mock is created using the method \srccode{mock} (imported
	from \srccode{org.mockito.Mockito}).

	\item The method \srccode{toString} is stubbed (using the statement
	\srccode{when}).
	\begin{itemize}
		\item It will throw a \srccode{NullPointerException} (as defined by the
		statement \srccode{thenThrow}).
	\end{itemize}
\end{itemize}
\end{block}
\end{frame}
