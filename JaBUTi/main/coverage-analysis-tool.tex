\begin{frame}[c,parent={cmap:jabuti-software-testing},hasnext=true,hasprev=false]
\label{cmap:coverage-analysis-tool}
\frametitle{Coverage analysis tool}

\insertcmap{Courses-SoftwareTesting-JaBUTi-JaBUTiToolsCoverageAnalysis}
\end{frame}


\begin{frame}[parent={cmap:coverage-analysis-tool},hasnext=true,hasprev=true]
\frametitle{Coverage analysis tool}
\label{concept:coverage-analysis-tool}
\label{concept:software-testing-tool}

\begin{block:concept}{}
JaBUTi's main tool is the coverage analysis tool. It uses test
criteria from the structural test technique to create test requirements
that a given test set must satisfy.
\end{block:concept}

\begin{block:procedure}{Basic workflow for coverage analysis}
\begin{enumerate}
	\item selection of classes to be tested;

	\item test requirements creation;

	\item visualization of test requirements, definition-use graph, and
	related bytecode and source code;

	\item instrumentation of classes to be tested;

	\item specification and execution of test cases;

	\item trace collection and coverage calculation;

	\item test case management;

	\item identification of infeasible test cases (test requirement
	management).
\end{enumerate}
\end{block:procedure}

\hfill
\refie{example:jabuti-new-project}{\beamerbutton{Example: Create a new project}}
\refie{example:jabuti-import}{\beamerbutton{Example: Import and run JUnit test set}}
\end{frame}

