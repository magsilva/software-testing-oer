\begin{frame}[c,parent={cmap:jabuti-software-testing},hasnext=true,hasprev=false]
\label{cmap:jabuti-test-project}
\label{cmap:test-project}
\frametitle{Test project}

\insertcmap{Courses-SoftwareTesting-JaBUTi-JaBUTiTestProject}
\end{frame}


\begin{frame}[parent={cmap:jabuti-test-project},hasnext=true,hasprev=true]
\label{concept:test-project}
\frametitle{Test project}

\begin{block:fact}{Test session $\rightarrow$ Test project}
\begin{itemize}
	\item A test session in JaBUTi is configured by creating a test project.
\end{itemize}
\end{block:fact}

\begin{block:concept}{Test project}
A test project is characterized by a file with the necessary information
about the application under testing:
\begin{itemize}
	\item<3-> base class file,
	\item<4-> complete set of classes required by the base class,
	\item<5-> set of classes to be instrumented (and tested),
	\item<6-> set of classes that should not be instrumented (and tested),
	\item<7-> \srccode{CLASSPATH} environment variable necessary to run
	the base class.
\end{itemize}
\end{block:concept}
\end{frame}



\begin{frame}
\frametitle{Test project}

\begin{block:concept}{Test project}
\begin{itemize}
	\item<1-> It also stores some information generated by the tool:
	\begin{itemize}
		\item test requirements (for every criteria supported by JaBUTi),

		\item test cases execution results (test case name, and status).
	\end{itemize}

	\item<2-> Those information are saved to a file which extension is
	\srccode{.jbt}.
	\begin{itemize}
		\item This file is an XML document.
	\end{itemize}
\end{itemize}
\end{block:concept}
\end{frame}
