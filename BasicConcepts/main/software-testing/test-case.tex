\begin{frame}[parent={cmap:software-testing-foundations}, hasprev=false, hasnext=true]
\frametitle{Test case}
\label{concept:test-case}
\label{concept:input-domain}
\label{concept:output-domain}
\label{concept:input-data}
\label{concept:output-data}

\begin{block:concept}{Simplified definition}
A test case is a pair consisting of test data (a set of values, one for each
input variable) to be input to the program and the expected output.
\end{block:concept}


\begin{block:concept}{A better definition}
A test case is usually defined as a tuple $(d, S(d))$, where:
\begin{itemize}
	\item $d \in D$ (and $D$ is the input domain), and
	\item $S(d)$ represents the expected output for the input $d$
	according to specification $S$.
\end{itemize}
\end{block:concept}

\hfill
\refie{example:sort-test-cases}{\beamerbutton{Example: Test cases for a sort method}}
\refie{example:num-zero-test-cases}{\beamerbutton{Example: Test cases for the numZero method}}
\end{frame}


\begin{frame}[hasprev=false, hasnext=true]
\frametitle{Test case}
\framesubtitle{Assessing a test case}
\label{concept:test-case-success}
\label{concept:test-case-failure}

\begin{block:fact}{Successful test cases}
\begin{itemize}
	\item At first, a well-constructed and executed test case is successful
	when it finds errors~\cite[p. 7]{myers:2004}

	\item It is also successful when it eventually establishes that there are
	no more errors to be found (as when applying a test criteria and satisfying
	all the test requirements).
\end{itemize}
\end{block:fact}

\begin{block:fact}{Unsuccessful test cases}
\begin{itemize}
	\item A unsuccessful test case is one that causes a program to product the
	correct result without finding any error.
\end{itemize}
\end{block:fact}

\hfill
\refie{example:doctor-laboratory-test}{\beamerbutton{Analogy for test cases success and failure}}
\end{frame}



\begin{frame}
\frametitle{Test case}

\begin{block:fact}{Execution order}
\begin{itemize}
    \item There are two styles of test case design regarding order of test
    execution:
	\begin{itemize}
		\item cascading test cases, and
		\item independent test cases.
	\end{itemize}
\end{itemize}
\end{block:fact}
\end{frame}


\begin{frame}
\label{concept:cascading-test-case}
\frametitle{Test case}
\framesubtitle{Cascading test case}

\begin{block:concept}{Definition}
Cascading test cases are test cases that build on each other.
\end{block:concept}


\begin{block:fact}{Advantages and disadvantages}
\begin{itemize}
	\item The advantage of cascading test cases is that each test case is
	typically small and simple.

	\item The disadvantage of cascading test cases is that if one test fails,
	the subsequent tests may be invalid.
\end{itemize}
\end{block:fact}

\hfill
\refie{example:cascading-test-case}{\beamerbutton{Example: Cascading test cases}}
\end{frame}



\begin{frame}[hasprev=true, hasnext=false]
\label{concept:independent-test-case}
\frametitle{Test case}
\framesubtitle{Independent test case}

\begin{block:concept}{Definition}
Independent test cases are entirely self contained.
\begin{itemize}
	\item Independent test cases neither build on each other nor require that
	other tests have been successfully executed.
\end{itemize}
\end{block:concept}

\begin{block:fact}{Advantages and disadvantages}
\begin{itemize}
	\item The advantage of independent test cases is that any number of tests
	can be executed in any order.

	\item The disadvantage of independent test cases is that each test tends to
	be larger and more complex and thus more difficult to design, create, and
	maintain.
\end{itemize}
\end{block:fact}
\end{frame}