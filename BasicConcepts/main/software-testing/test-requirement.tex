\begin{frame}[parent={cmap:software-testing-foundations}, hasprev=false, hasnext=true]
\frametitle{Test requirement}
\label{concept:test-requirement}

\begin{block:concept}{Definition}
A test requirement is a specific element of a software artifact that a test
case must satisfy or cover.
\end{block:concept}

\begin{block:fact}{How a test requirement is created?}
\begin{itemize}
	\item Test requirements are derived from the program under testing using
	a specific test criterion.
\end{itemize}
\end{block:fact}

\begin{block:fact}{What are test requirements for?}
\begin{itemize}
	\item A test requirement can:
	\begin{itemize}
		\item evaluate a test set, and
		\item generate a test set.
	\end{itemize}
\end{itemize}
\end{block:fact}
\end{frame}


\begin{frame}[hasprev=true, hasnext=false]
\label{concept:test-set}
\label{concept:c-adequate-test-set}
\frametitle{Test requirement}
\framesubtitle{Test set}

\begin{block:concept}{Definition}
A test set is a set of test cases.
\end{block:concept}

\begin{block:fact}{Test sets and test requirements}
\begin{itemize}
	\item A test set can be improved by adding test cases that exercise
	uncovered requirements.
	\begin{itemize}
		\item The best test set is the smallest one that indicates the largest
		set of faults.
	\end{itemize}
\end{itemize}
\end{block:fact}

\begin{block:concept}{C-adequate test sets}
\begin{itemize}
	\item When a test set $T$ satisfies all the test requirements derived from
	a program using a given criterion $C$, $T$ is said $C-adequate$.
\end{itemize}
\end{block:concept}
\end{frame}