\begin{frame}[parent={cmap:software-testing-foundations}, hasprev=false, hasnext=true]
\frametitle{Test technique}
\label{concept:test-technique}

\begin{block:concept}{Definition}
Test techniques are types of testing defined according to the
source of information used to carried out the testing activity.
\end{block:concept}

\begin{block:fact}{Test techniques and test criteria}
\begin{itemize}
    \item Each test technique has a set of associated test criteria.
\end{itemize}
\end{block:fact}
\end{frame}



\begin{frame}[hasprev=true, hasnext=true]
\frametitle{Test technique}

\begin{block:fact}{Software test techniques}
\begin{itemize}
	\item Exhaustive testing

	\item Random testing

	\item Partition testing
	\begin{itemize}
		\item Fault-based testing

		\item Functional testing

		\item Structural testing
	\end{itemize}
\end{itemize}
\end{block:fact}
\end{frame}



\begin{frame}
\frametitle{Test technique}
\framesubtitle{Exhaustive testing}
\label{concept:exhaustive-testing}

\begin{block:concept}{Definition}
An exhaustive test executes the software with all possible value from its
input domains.
\end{block:concept}

\begin{figure}
    \centering
    \includegraphics[width=\textwidth]{figs/exhaustive-software-testing}
\end{figure}

\hfill
\refie{example:blech-exhaustive-testing}{\beamerbutton{Example: Exhaustive testing of the blech function}}
\end{frame}



\begin{frame}
\frametitle{Test technique}
\framesubtitle{Exhaustive testing}

\begin{block:fact}{Exhaustive testing limitations}
\begin{itemize}
	\item Can be prohibitive due to time and cost constraints for finite
	but large input domain.

	\item Impossible if the input domain is infinite.

	\item Infeasible in general.
\end{itemize}
\end{block:fact}
\end{frame}



\begin{frame}
\frametitle{Test technique}
\framesubtitle{Random testing}
\label{concept:random-testing}

\begin{block:concept}{Definition}
Random testing uses a systematic method to generate test cases: it
requires modeling the input space and then sampling data from the input
space randomly.
\end{block:concept}

\begin{figure}
    \centering
    \includegraphics[width=\textwidth]{figs/random-software-testing}
\end{figure}
\end{frame}


\begin{frame}
\frametitle{Test technique}
\framesubtitle{Random testing}

\begin{block:concept}{Reliability}
\begin{itemize}
	\item Using random testing, statistical measures of reliability can be
	achieved according to an operational profile.

	\item For every (input data of a) test case, it is assigned a probability
	distribution according to their occurrence in actual operation.
\end{itemize}
\end{block:concept}

\begin{block:fact}{Effectiveness}
\begin{itemize}
	\item It depends on the correctly definition of the operational profile.

	\item If the probability of occurrence of each input data is the same,
	random testing is regarded as the least effective technique for software
	testing~\cite[p. 43]{myers:2004}.
\end{itemize}

\end{block:fact}


\end{frame}


\begin{frame}
\frametitle{Test technique}
\framesubtitle{Partition testing}
\label{concept:partition-testing}

\begin{block:concept}{Definition}
Partition testing is meant any testing scheme which forces execution
of at least one test case from each subset of a partition of the input
domain
\end{block:concept}

\begin{figure}
    \centering
    \includegraphics[width=\textwidth]{figs/partition-software-testing}
\end{figure}
\end{frame}



\begin{frame}
\frametitle{Test technique}
\framesubtitle{Functional testing}
\label{concept:functional-testing}

\begin{block:concept}{Definition}
Functional testing is a technique based solely on the requirements and
specifications.
\end{block:concept}

\begin{block:fact}{}
\begin{itemize}
	\item Functional testing is also known as black box testing.

	\item Functional testing obtains test requirements from the
	software specification.
	\begin{itemize}
		\item Functional testing requires no knowledge of the internal paths,
		structure, or implementation of the software under test.
	\end{itemize}
\end{itemize}
\end{block:fact}

\hfill
\refie{example:functional-testing}{\beamerbutton{Example}}
\end{frame}


\begin{frame}
\frametitle{Test technique}
\framesubtitle{Functional testing}

\begin{block:fact}{Functional test criteria}
\begin{itemize}
	\item Equivalence partition.
	\item Boundary-value analysis.
	\item Cause-effect graph.
	\item \ldots
\end{itemize}
\end{block:fact}
\end{frame}



\begin{frame}
\frametitle{Test technique}
\framesubtitle{Structural testing}

\begin{block:concept}{Definition}
Structural testing is a technique based on the internal paths, structure,
and implementation of the software under test.
\end{block:concept}


\begin{block:fact}{}
\begin{itemize}
	\item Structural testing is also known as white box testing.

	\item Structural testing obtains test requirements from implementation
	features.
\end{itemize}
\end{block:fact}

\begin{figure}
    \centering
    \includegraphics[width=6cm]{figs/structural-testing}
\end{figure}
\end{frame}


\begin{frame}
\frametitle{Test technique}
\framesubtitle{Structural testing}
\label{concept:structural-testing-criteria}

\begin{block:fact}{Control-flow based criteria}
\begin{itemize}
	\item Criteria based on the flow of control within a program:
	\begin{itemize}
		\item Statement coverage.
		\item Decision coverage.
		\item Condition coverage.
		\item \ldots
	\end{itemize}
\end{itemize}
\end{block:fact}


\begin{block:fact}{Data-flow based criteria}
\begin{itemize}
	\item Criteria based on the usage of data (variable creation, definition,
	and use):
	\begin{itemize}
		\item All-uses.
		\item All-potential-uses
		\item \ldots
	\end{itemize}
\end{itemize}
\end{block:fact}


\hfill
\refie{example:structural-testing}{\beamerbutton{Example}}
\end{frame}




\begin{frame}
\frametitle{Test technique}
\framesubtitle{Fault-based testing}
\label{concept:fault-based-testing}

\begin{block:concept}{Definition}
Fault-based testing is a technique in which testing is based on
historical information about common faults detected during the software
development life cycle.
\end{block:concept}

\begin{figure}
    \centering
    \includegraphics[width=7cm]{figs/mutation-testing}
\end{figure}
\end{frame}


\begin{frame}[hasprev=true, hasnext=false]
\frametitle{Test technique}
\framesubtitle{Fault-based testing}
\label{concept:fault-based-test-criteria}

\begin{block:fact}{Fault-based test criteria}
\begin{itemize}
	\item Error seeding.
	\item Mutation:
	\begin{itemize}
		\item Mutation analysis.
		\item Interface mutation.
		\item \ldots
	\end{itemize}
\end{itemize}
\end{block:fact}

\hfill
\refie{example:fault-based-testing}{\beamerbutton{Example}}
\end{frame}
